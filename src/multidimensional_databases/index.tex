% - Explain and discuss multidimensional concepts incl.
% 	- cubes, dimensions, facts, and measures, including complex issues and solutions
% 	- relational representations
% 	- analysis and querying
% - Describe how they are used in your mini project
% 	- Choice and granularity of business process
% 	- Dimensional modeling
% 	- Star/snowflake schema
% 	- Analysis of your F-Klub’s DW

% Remember that there were several lectures about this. Not just lecture 1.

\section{Multidimensional Databases}
\begin{multicols}{2}
\begin{itemize}
\item
  Multidimensional refers to a way of organizing and analyzing data.
\item
  Here, data is stored in a multi-layered, hierarchical structure.
\item
  \textbf{Multidimensional models} categorize data as being either facts
  or dimensions

  \begin{itemize}
    \item
    \textbf{Facts}, with associated numerical \emph{measures}, or

    \begin{itemize}
        \item
      Facts are the important entity, e.g.~a sale
    \item
      Facts live in a multidimensional \textbf{cube}.
    \end{itemize}
  \item
    \textbf{Measures} represent the fact property, e.g.~sales price

    \begin{itemize}
        \item
      They have two components: the numerical value and an aggregation
      formula (e.g.~SUM).
    \end{itemize}
  \item
    \textbf{Dimensions} that describe facts. E.g. a sale with the
    dimensions Product, Store, and Time.

    \begin{itemize}
        \item
      Used for \textbf{selection} of data and \textbf{grouping} of data
    \item
      Consist of dimension values: e.g.~Product can have ``milk'',
      ``cream'', \ldots, and time could have ``1/1/2001'', ``2/1/2001'',
      \ldots{}
    \item
      They have hierarchies with levels. Typically 3-5 levels.

      \begin{itemize}
            \item
        \textbf{Time}: Day → Month → Quarter → Year → T \textbar{}
        (written bottom → top)
      \end{itemize}
    \item
      Levels can have attributes. E.g. Day has Workday
    \end{itemize}
  \item
    \textbf{Cubes} are where the data is organized into dimensions and
    facts

    \begin{itemize}
        \item
      Cubes consist of cells, which is a given combination of dimension
      values (or empty).
    \item
      Can have many dimensions. \textgreater{} 3 then it can be called a
      hypercube.

      \begin{itemize}
            \item
        There's no limit, but it's usually 4-12.
      \end{itemize}
    \end{itemize}
  \end{itemize}
\item
  \textbf{Complex issues: dimensions changing over time}

  \begin{itemize}
    \item
    \textbf{Slowly changing dimensions:} infrequent changes. Mostly
    regarding attribute values, but could be about adding more
    attributes to the dimensions.

    \begin{itemize}
        \item
      E.g. store changing size, products changing descriptions, etc.
    \item
      \textbf{Solutions}

      \begin{itemize}
            \item
        Don't handle it; can lead to overwriting old values so losing
        historical data.
      \item
        Version dimension values (e.g.~timestamps/create new)---leads to
        larger dimension tables
      \item
        Capture previous and current (e.g.~two versions of attribute),
        but then you don't know when the change happened.
      \item
        Split into changing and constant attribute.
      \end{itemize}
    \end{itemize}
  \item
    \textbf{Rapidly changing dimensions:} dimensions change rapidly.

    \begin{itemize}
        \item
      \textbf{Solutions}: 2 and 3 are here. Dimension splitting (4) is
      good.
    \item
      So look at frequently changing attributes and separate those from
      those that don't change as often. Then create two new
      `minidimensions'.
    \item
      Then you can even optimize further: frequently changing
      \emph{numeric} attributes can become discrete / banded values
      (within a range). That leads to info loss, but that often happens
      during analysis anyway.
    \item
      To maintain relationship between the new minidimensions, can add
      surrogate keys for both into the fact table---so each new entry in
      fact table you also get the surrogate key for each new dimension.
    \end{itemize}
  \end{itemize}
\item
  \textbf{Relational representations}---how do we represent
  multidimensional model with relational databases?

  \begin{itemize}
    \item
    We do many-to-one relationships from facts to dimension values
  \item
    And many-to-one relationships from lower to higher levels in the
    hierarchies
  \item
    We use \textbf{star schemas} or \textbf{snowflake schemas} with fact
    tables and dimension tables.
  \item
    \textbf{Fact tables} store facts. One row for each. One column for
    each measure. One column for each dimension, with a foreign key to
    dimension table.
  \item
    \textbf{Dimension tables} store dimension data.
  \item
    \textbf{Star schema}

    \begin{itemize}
        \item
      Has one fact table, for each business process.
    \item
      Has one de-normalized dimension table for each dimension with one
      column for each level and attribute (except T).
    \item
      It's simple and easy to use, but hierarchies are hidden in the
      columns, which hides the hierarchies. It has good performance, as
      it's recognized by many RDBMSes.
    \end{itemize}
  \item
    \textbf{Snowflake schema}

    \begin{itemize}
        \item
      Dimensions are normalized (broken into multiple related tables).

      \begin{itemize}
            \item
        Helps reduce data redundancy \& improves integrity.
      \end{itemize}
    \item
      One dimension table per level.
    \item
      Each dimension table has an integer key, a level name, one column
      per attribute, and a foreign key to the next level.
    \item
      Here, hierarchies are made explicit, but it has worse performance
      and is harder to use due to many joins.
    \end{itemize}
  \end{itemize}
\item
  \textbf{OLAP Queries}

  \begin{itemize}
    \item
    There are two kinds of queries:
  \item
    \textbf{Navigation queries}, which examine one dimension:

    \begin{itemize}
        \item
      \texttt{SELECT\ DISTINCT\ l\ from\ d\ {[}WHERE\ p{]}}
    \end{itemize}
  \item
    \textbf{Aggregation queries} that summarize fact data

    \begin{itemize}
        \item
      \texttt{SELECT\ d1.l1,\ d2.l2,\ SUM(f.m)\ FROM\ d1,\ d2,\ f\ WHERE\ f.dk1\ =\ d1.dk1\ AND\ f.dk2\ =\ d2.dk2\ {[}AND\ p{]}\ GROUPO\ BY\ d1.l1,\ d2.l2}
    \end{itemize}
  \end{itemize}
\item
  \textbf{Our project} focused on the business process of
  \textbf{sales}.

  \begin{itemize}
    \item
    \textbf{Two relevant questions:}

    \begin{itemize}
        \item
      Which venue has the highest gross income?

      \begin{itemize}
            \item
        To determine which can get more resources (new fridge, etc.) or
        which to perhaps discontinue
      \end{itemize}
    \item
      Which product has the highest gross income?

      \begin{itemize}
            \item
        To determine wheter it would make sense to buy more /
        discontinue a certain product
      \end{itemize}
    \end{itemize}
  \item
    \textbf{Dimensional modeling}

    \begin{itemize}
        \item
      Fact: measure is price of sale

      \begin{itemize}
            \item
        Measure is additive across all dimensions as it's possible to
        meaningfully aggregate across them
      \item
        Granularity: sales per hour per product name per venue name
      \end{itemize}
    \item
      Dimensions: we want to consider when, what, and where. So
      \textbf{product}, \textbf{time}, and \textbf{venue}.

      \begin{itemize}
            \item
        \textbf{Product:} T → Category → Name

        \begin{itemize}
                \item
          So we can query for all products, categories, or name
        \item
          \textbf{Product table}: Id (surrogate key), Name, Category
        \end{itemize}
      \item
        \textbf{Time:} T → Year → Month → Day → Hour

        \begin{itemize}
                \item
          and season only accessible to T and Hour
        \item
          \textbf{Time table:} Id (surrogate key), year, month, day,
          hour, day name, season
        \end{itemize}
      \item
        \textbf{Venue:} T → Type → Name

        \begin{itemize}
                \item
          \textbf{Venue table:} Id (surrogate key), name, type
        \end{itemize}
      \end{itemize}
    \end{itemize}
  \item
    We used \textbf{star schema} as we wanted to aggregate information
    and didn't want to worry about details for individual transactions.
  \item
    \textbf{Analysis of case}---May was most selling month. May be
    because semester projects finishing at that time (people
    celebrate?). Best selling venue was web sales.
  \end{itemize}
\end{itemize}
\end{multicols}