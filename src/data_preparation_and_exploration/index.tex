% - Data Preparation: general concepts and process
% 	- Transformer and estimator;
% 	- Data cleaning: deal with missing, duplicate, and invalid data;
% 	- Data preparation for different kinds of data: continuous data, categorical data, text data
% 	- Feature extraction (PCA) and selection (Chi-Square)
% - Data Exploration:
% 	- Univariate analysis
% 		- Statistics, histogram, box/violin plot, bar/pie chart
% 	- Multivariate analysis
% 		- Correlation matrix, Scatter plot

\section{Data Preparation and Exploration}
\begin{multicols}{2}
\begin{itemize}
\item
  Data needs to be prepared according to the tools you will use on them.
\item
  Preparing data involves cleaning, extracting and selecting useful
  features.
\item
  Apache Spark\textquotesingle s MLLib provides API to help with this
  preparation process (and further!)
\item
  Some central concepts are \textbf{transformers and estimators}.

  \begin{itemize}
    \item
    \textbf{Transformers} help us format data to a useful format. With
    MLLib, just call \texttt{transformer.transform(df)}.

    \begin{itemize}
        \item
      Give example?
    \end{itemize}
  \item
    \textbf{Estimators} can also help us prepare input data.
    They\textquotesingle re similar to transformers, but need to
    \textbf{fit} the data first (initialization), and then
    \textbf{transform}.

    \begin{itemize}
        \item
      For example, if you need to normalize data e.g. with the
      \texttt{StandardScaler} which uses the mean and standard
      deviation, you need to do it over multiple passes...
      can\textquotesingle t calculate the mean as you are transforming.
    \item
      I.e. it computes values that are preliminary to do the scaling
      computation during fitting, and then does the scaling computation
      during transformation.
    \end{itemize}
  \end{itemize}
\item
  \textbf{Data cleaning} is necessary because data is usually dirty. So
  how do we do it?

  \begin{itemize}
    \item
    \textbf{Remove missing data}---e.g. removing rows with missing data
    (nulls). E.g. \texttt{dropna()}.
  \item
    \textbf{Impute missing data}---replacing / inserting values where
    there are none. E.g. \texttt{fillna()} to use fixed value, or
    \texttt{imputer} to use mean/median/mode.
  \item
    \textbf{Handling duplicate data}---deleting duplicate rows
    (full/partial), or merging (partial) duplicates. E.g.
    \texttt{dropDuplicate()}.
  \item
    \textbf{Handle invalid data}---need either external data source or
    domain knowledge/reasoning to fix poor/wrong data.
  \end{itemize}
\item
  Now to how we can \textbf{represent data or labels}

  \begin{itemize}
    \item
    \textbf{Continuous features}---numbers from {\(- \infty\)} to
    {\(\infty\)}.

    \begin{itemize}
        \item
      Usually don\textquotesingle t deal with it, but can use
    \item
      \textbf{Bucketing}---converting to categorical features using
      thresholds.
    \item
      \textbf{Normalization}---which is important if various cont.
      features use different scales.
    \end{itemize}
  \item
    \textbf{Categorical features}---these are discrete numbers, but we
    need to make them actual numbers so our algorithms can work with
    them.

    \begin{itemize}
        \item
      Commonly use indexing to convert categorical feature to numerical.
    \item
      E.g. the \texttt{OneHotEncoder}.
    \end{itemize}
  \item
    \textbf{Text data}---mostly focusing on free-form text, as string
    categorical vars are handled above. Here are the primary operations:

    \begin{itemize}
        \item
      \textbf{Cleaning text:} tokenization (\& removing stop words,
      finding word combinations, converting text to data)

      \begin{itemize}
            \item
        \textbf{Tokenization}: MLLib provides transformer
        \texttt{Tokenizer} and \texttt{regexTokenizer} to do this.

        \begin{itemize}
                \item
          \texttt{Tokenizer} splits by whitespace.
        \item
          \texttt{regexTokenizer} splits based on regex.
        \end{itemize}
      \item
        \textbf{Removing stop words}, i.e. "the" and "and".

        \begin{itemize}
                \item
          \texttt{StopWordsRemover} is a transformer provided by MLLib.
        \end{itemize}
      \item
        \textbf{Word combination}: in a sentence, combinations of words
        usually makes more sense for analysis that a set of individual
        words. "big apple" says more than "big", and "apple".

        \begin{itemize}
                \item
          MLLib provides transformer \texttt{NGram}. Usually used after
          tokenization and/or removing stop words.
        \item
          Basically that would convert the tokens into pairs of {\(n\)}
          tokens
        \end{itemize}
      \end{itemize}
    \item
      \textbf{Converting to number data:} word2vec (advanced)

      \begin{itemize}
            \item
        Now we need to convert text to numerical features. MLLib
        provides \texttt{CountVectorizer} and \texttt{Word2Vec}.
      \item
        \textbf{Word2Vec} is a deep learning-based tool to compute a
        vector representation for a set of words.

        \begin{itemize}
                \item
          It\textquotesingle s a blackbox representation, so
          representation rule isn\textquotesingle t deterministic, but
          has superior performance in many NLP tasks.
        \item
          Using w2v, a neural net will be trained on the dataset.
        \item
          MLLib provides estimator for this (fit then transform).
        \end{itemize}
      \end{itemize}
    \end{itemize}
  \end{itemize}
\item
  \textbf{Feature extraction}---techniques to derive \emph{new} features
  from raw data.

  \begin{itemize}
    \item
    \textbf{Principal Component Analysis
    (PCA)}---dimensionality-reduction method, reducing number of
    features in a dataset.

    \begin{itemize}
        \item
      Goal: Simplify data without losing too much information. For
      example, you can merge several correlated features into one.
    \item
      MLLib has estimator.
    \end{itemize}
  \end{itemize}
\item
  \textbf{Feature selection}---to evaluate and select important features
  (not deriving new ones).

  \begin{itemize}
    \item
    \textbf{Chi-Square}---test \emph{independence hypothesis} (features
    and labels are independent) by checking p-value. If under threshold,
    assume it\textquotesingle s false---so we think features and labels
    are dependent (good because then we can use feature to predict
    label).
  \end{itemize}
\item
  \textbf{Data exploration}

  \begin{itemize}
    \item
    \textbf{Univariate analysis}---data exploration with only one
    variable

    \begin{itemize}
        \item
      \textbf{Statistics}---seeing e.g. count, mean, stddev, 25\%, min,
      etc.
    \item
      \textbf{Histogram}---approximation of distribution of numerical
      data. First bucket (divide entire range of values into series of
      bins), then count how many values fall into each range.

      \begin{itemize}
            \item
        Used for continuous data. Seeing rectangles of adjacent bins
        touching each other indicate the original variable is
        continuous.
      \end{itemize}
    \item
      \textbf{Box/violin plot}---describes data distribution over
      quartiles

      \begin{itemize}
            \item
        Similar to box plots, but also show probability desity of data
        at different values. Used for continuous data.
      \end{itemize}
    \item
      \textbf{Bar/pie chart}---presents categorical data with
      rectangular bars.

      \begin{itemize}
            \item
        And pie charts also presents proportions.
      \end{itemize}
    \end{itemize}
  \item
    \textbf{Multivariate analysis}---data exploration with multiple
    variables

    \begin{itemize}
        \item
      \textbf{Correlation matrix}---table of correlation coefficients
      betwen all possible pairs of values.

      \begin{itemize}
            \item
        Between {\(- 1\)} and {\(1\)}
      \item
        Positive/negative to indicate correlation (pos/neg corr)
      \item
        Zero means 0 correlation
      \item
        Only for continuous (?)
      \end{itemize}
    \item
      \textbf{Scatter plot}---represent values for two different numeric
      variables. Helps observe relationships.
    \item
      \textbf{Line chart}---represents relationship between a
      categorical and continuous variable. (E.g. day of week v. sales
      volume)
    \end{itemize}
  \end{itemize}
\end{itemize}
\end{multicols}