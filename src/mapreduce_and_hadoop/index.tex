% - Explain and discuss MapReduce/Hadoop incl.
% 	- what MapReduce/Hadoop is needed for
% 	- the map and reduce functions
% 	- an example of a MapReduce program
% 	- how the framework works: how it handles important issues
%
% Remember that you should be able to write some small MR programs (using pseudo code).


\section{MapReduce and Hadoop}
\begin{multicols}{2}
\begin{itemize}
\item
  2 ways to scale, horizontal and vertical. Vertical is fast but
  limited. Horizontal is very fast but you need to deal with race
  conditions, deadlocks, node failures, etc.
\item
  For horizontal scaling, ideally, we want a system that scales linearly
  with data / size of cluster.
\item
  These requirements were the motivation for \textbf{MapReduce}.

  \begin{itemize}
    \item
    MR handles partitioning, scheduling, failures, etc. So users can
    focus on computations.
  \item
    But it's made by Google and isn't open source, so Apache Hadoop's
    MapReduce is a popular alternative.
  \end{itemize}
\item
  \textbf{Map} and \textbf{Reduce}

  \begin{itemize}
    \item
    With MR, you only need to specify 2 functions: \textbf{map} and
    \textbf{reduce}.
  \item
    The two functions take key/value pairs as input and produces another
    set of key/value pairs as output.
  \item
    \textbf{Map} takes an input pair and produces a set of intermediate
    key/value pairs.

    \begin{itemize}
        \item
      The system then groups all intermediate pairs with the same key.
    \end{itemize}
  \item
    \textbf{Reduce} takes these keys and the accompanying list of values
    and merges the values to form a smaller set (often empty or with a
    single value, e.g.~summing/averaging).
  \end{itemize}
\item
  \textbf{Example: Word Count}
\end{itemize}

\begin{verbatim}
# key is e.g. doc name; value is doc content
map(String key, String value):
    foreach word in value:
        EmitIntermediate(word, 1)

reduce(String key, Iterator<int> values):
    int result = 0;
    foreach v in values:
        result += v;
 
    Emit(result);
\end{verbatim}

\begin{itemize}
\item
  \textbf{How the framework works}

  \begin{itemize}
    \item
    Map invocations are distributed across many machines---they run
    concurrently.
  \item
    To facilitate this, input is \textbf{partitioned} in logical splits
    so it can be processed in parallel.

    \begin{itemize}
        \item
      Input data is stored in distributed file system
    \end{itemize}
  \item
    MapReduce system tries to schedule the map task to the node where
    the data is located, which is called \textbf{data locality}.
  \item
    Intermediate pairs produced by map are also partitioned
  \item
    Reduce invocations can also be distributed, but doesn't run until
    all map tasks are done.

    \begin{itemize}
        \item
      Intermediate results with same key should be sent to same reducer.
    \end{itemize}
  \end{itemize}
\item
  \textbf{Hadoop} uses \textbf{YARN} to handle the issues mentioned
  earlier

  \begin{itemize}
    \item
    YARN = Yet Another Resource Negotiator.
  \item
    YARN is used for cluster resource management.
  \item
    It has a resource manager per cluster and a node manager per node.
  \item
    Uses containers to run processes, each having some memory and CPU.
  \item
    \textbf{When resources are requested}, locality constraints can be
    given---such as \emph{Data locality}, which means that jobs should
    be run in the same location as the data (seen earlier).

    \begin{itemize}
        \item
      If that isn't possible, they should at least run in the same
      server rack to avoid network bottlenecks.
    \item
      If YARN cannot find a good resource immediately, it can wait for
      resources to be freed.
    \end{itemize}
  \item
    \textbf{Resource assignment:}

    \begin{itemize}
        \item
      The job is submitted to YARN, which creates a container and starts
      the application master for MapReduce.
    \item
      That requests a container for each map task.

      \begin{itemize}
            \item
        When 5\% has finished, the application master starts to request
        containers for reduce tasks, so they're ready when all are done.
      \item
        Container runs the task in dedicated JVM (Java VM) for isolation
        of failures.
      \item
        Tasks report status and progress to application master.
      \end{itemize}
    \end{itemize}
  \end{itemize}
\end{itemize}
\end{multicols}