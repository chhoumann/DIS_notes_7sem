% - R-tree basis
% 	- Structure: MBR, internal node/entry, leaf node/entry
% - R-tree insertion
% 	- Overall procedure of insertions
% 	- Split and enlargement
% - R-tree deletion
% 	- Overall procedure of deletion
% 	- MBR shrinking and node underflow
% - Range query
% 	- Processing via R-tree
% - Nearest neighbor query
% 	- Defininition
% 	- Depth-first (branch-and-bound) search via R-tree
% 		- Distance metrics: mindist, minmaxdist
% 	- Best-first search via R-tree

\section{Spatial Data Indexing and Queries}
\begin{multicols}{2}
\begin{itemize}
\item
  \textbf{What is spatial data?}

  \begin{itemize}
    \item
    Structured data is e.g.~a row for each transaction. It's based on a
    schema.
  \item
    But that isn't good for maintaining spatial information. It would be
    too hard to answer questions regarding spatial topics (distance,
    etc.).
  \item
    Spatial data is how we store local data or shape information. Stored
    in spatial DB, which is a normal DB with enhancements to represent
    spatial objects (lines, points, polygons, etc.).
  \item
    The data is organized with indexing, which we'll get to. This allows
    for efficient access to spatial objects.
  \item
    \textbf{Spatial objects} hold spatial information (e.g.~location,
    geometry). Three types:

    \begin{itemize}
        \item
      \textbf{Points}: Dimensionality isn't important, only location.
      E.g. cities on a small-scale map.
    \item
      \textbf{Line segments and polylines}: objects whose lengths and
      locations are of importance. E.g. roads \& rivers.
    \item
      \textbf{Polygons}: Objects whose shapes, locations, and areas
      matter. E.g. forests, districts, a city area, etc.
    \end{itemize}
  \item
    \textbf{Spatial relation} is a dataset, which is a collection of
    spatial objects of the same type.
  \end{itemize}
\item
  \textbf{R-Trees}---data structure used to index spatial data.

  \begin{itemize}
    \item
    Need to understand \textbf{minimum bounding rectangles (MBRs)}
    first.

    \begin{itemize}
        \item
      Rectangles that have minimum area to cover a spatial object.
    \end{itemize}
  \item
    \textbf{Root} contains pointer to largest region in spatial domain.
  \item
    \textbf{Internal nodes} contain info on MBRs at that level +
    pointers to child nodes with regions overlapping the parent node's.
  \item
    \textbf{Leaf nodes} contain MBR info + object ID telling where
    objects are stored in memory.
  \end{itemize}
\item
  \textbf{R-Tree Insertion}

  \begin{enumerate}
  \def\labelenumi{\arabic{enumi}.}
    \item
    Start at root and go down to ``best-fit'' leaf node \(L\).

    \begin{itemize}
        \item
      Best fit = child whose area needs the least area enlargement to
      fit the object. Resolve ties by going to child with smallest area.
    \end{itemize}
  \item
    If the best-fit leaf \(L\) has space, insert entry (object) and
    stop.
  \item
    Else, split \(L\) into \(L_{1}\) and \(L_{2}\).

    \begin{enumerate}
    \def\labelenumii{\arabic{enumii}.}
        \item
      Adjust entry for \(L\) in its parent so the MBR now covers
      \(L_{1}\) only.
    \item
      Add entry in parent node of \(L\) for \(L_{2}\). (this could then
      cause the parent node to recursively split).
    \end{enumerate}
  \item
    Propagate changes upwards.
  \end{enumerate}

  \begin{itemize}
    \item
    \textbf{How to split:}

    \begin{itemize}
        \item
      \textbf{Criterion:} entries in node \(L\) plus newly inserted
      entry must be evenly distributed between \(L_{1}\) and \(L_{2}\).
    \item
      \textbf{Goal:} reduce likelihood of both \(L_{1}\) and \(L_{2}\)
      being searched on subsequent queries and to redistribute so as to
      minimize the sum of \(L_{1}\)'s area and \(L_{2}\)'s area.
    \item
      \textbf{Approaches:}

      \begin{itemize}
            \item
        \textbf{Exhaustive split}---test all groupings and choose best
        with least MBR enlargement.
      \item
        \textbf{Plane sweep}---move plane onto rectangles until 50\%
        objects are covered, then split according to coverage.
      \item
        \textbf{Linear/quadratic split}---choose two objects as seeds
        that are as far apart as possible. Then consider remaining
        objects in random order and assign it to the node requiring the
        smallest enlargement of its respective MBR.
      \end{itemize}
    \end{itemize}
  \end{itemize}
\item
  \textbf{R-Tree Deletion}

  \begin{enumerate}
  \def\labelenumi{\arabic{enumi}.}
    \item
    Find leaf node \(N\) that contains entry \(E\).
  \item
    Remove \(E\) from \(N\).
  \item
    If \(N\) goes underflow:

    \begin{enumerate}
    \def\labelenumii{\arabic{enumii}.}
        \item
      Eliminate \(N\) by removing entry \(E_{N}\) from parent node and
      add \(N\) to \(Q\) (set of eliminated nodes).
    \end{enumerate}
  \item
    Otherwise, shrink \(N\)'s MBR accordingly.
  \item
    Propagate upwards.

    \begin{enumerate}
    \def\labelenumii{\arabic{enumii}.}
        \item
      Reinsert all entries of nodes in \(Q\) into the tree using
      \texttt{Insert}.
    \end{enumerate}
  \end{enumerate}
\item
  Now to query the R-tree with spatial queries.
\item
  \textbf{Range query}---use query window to cover certain area. Find
  objects by looking at intersecting/contained objects.

  \begin{itemize}
    \item
    Knowing the size of the window, you can determien how far away an
    object is from another.
  \item
    \textbf{Search via R-Tree}---start at root node.

    \begin{itemize}
        \item
      If node isn't leaf, visit all MBR entries that intersect query
      \(W\). Recurse on each of those nodes.
    \item
      If node is a leaf, visit each object that intersects query \(W\).
      Test range query against exact geometry of the object. If still
      intersecting, report object.
    \end{itemize}
  \end{itemize}
\item
  \textbf{Nearest neighbor query}---to find the object nearest to a
  query point.

  \begin{itemize}
    \item
    \textbf{Straightforward solution}---just compare spatial object with
    all objects in spatial relation and keep track of the one with the
    least distance. Not a good solution.

    \begin{itemize}
        \item
      Better is to use MINDIST. Consider MBR \(R\) with a number of
      points, then \(MINDIST(p, R)\) is is the min distance between
      \(p\) and \(R\).
    \item
      Meaning, it's the distance of the closest point in \(R\).
    \item
      \textbf{Three cases:}

      \begin{itemize}
            \item
        \(p\) is inside \(R\), so \(MINDIST(p,R)=0\)
      \item
        \(p\) is beside of \(R\), so MINDIST is distance from \(p\) to
        closest side.
      \item
        \(p\) is in corner space of \(R\), so MINDIST is found using
        Pythagoras. Form triangle and calculate hypothenuse.
      \end{itemize}
    \end{itemize}
  \item
    \textbf{Best-first NN Search}---by checking the most possible
    candidates.

    \begin{itemize}
        \item
      Use priority queue to organize seen entries and to prioritize the
      next node to be visited.
    \item
      Always visits the closest one (from priority queue).
    \item
      When an object is seen from the priority queue, it's the NN.
    \end{itemize}
  \end{itemize}
\end{itemize}
\end{multicols}
